
%%%%%%%%%%%%%%%%%%%%%%% file typeinst.tex %%%%%%%%%%%%%%%%%%%%%%%%%
%
% This is the LaTeX source for the instructions to authors using
% the LaTeX document class 'llncs.cls' for contributions to
% the Lecture Notes in Computer Sciences series.
% http://www.springer.com/lncs       Springer Heidelberg 2006/05/04
%
% It may be used as a template for your own input - copy it
% to a new file with a new name and use it as the basis
% for your article.
%
% NB: the document class 'llncs' has its own and detailed documentation, see
% ftp://ftp.springer.de/data/pubftp/pub/tex/latex/llncs/latex2e/llncsdoc.pdf
%
%%%%%%%%%%%%%%%%%%%%%%%%%%%%%%%%%%%%%%%%%%%%%%%%%%%%%%%%%%%%%%%%%%%


\documentclass[runningheads,a4paper]{llncs}

\usepackage{amssymb}
\setcounter{tocdepth}{3}
\usepackage{graphicx}

\usepackage{url}
\urldef{\mailsa}\path|{caldwelt, david.t.coleman, nikolaus.correll}@colorado.edu|    
\newcommand{\keywords}[1]{\par\addvspace\baselineskip
\noindent\keywordname\enspace\ignorespaces#1}

\usepackage{color}
\usepackage{wrapfig}


\begin{document}

\mainmatter  % start of an individual contribution

% first the title is needed
\title{Robotic Manipulation for Identification of Flexible Objects}

% a short form should be given in case it is too long for the running head
\titlerunning{Robotic Manipulation for Identification of Flexible Objects}


\author{T. M. Caldwell and  D. Coleman and N. Correll%
\thanks{
This work was supported by a NASA
Early Career Faculty fellowship NNX12AQ47GS02. We are grateful for this support.}%
}
%
\authorrunning{Robotic Manipulation for Identification of Flexible Objects}
% (feature abused for this document to repeat the title also on left hand pages)

% the affiliations are given next; don't give your e-mail address
% unless you accept that it will be published
\institute{Department of Computer Science, University of Colorado,\\
1111 Engineering Dr, Boulder, CO 80309, USA\\
\mailsa}

\toctitle{Lecture Notes in Computer Science}
\tocauthor{Authors' Instructions}
\maketitle

\begin{abstract}
Blah
\end{abstract}

\section{Introduction}

\begin{figure}[!htb]
\centering
\def\svgwidth{.95\textwidth}%500pt}
\input{baxter_y.pdf_tex}
\caption{The physical experiment, vision capture (in green), and simulation of baxter manipulating a flexible object.} %The vision capture is shown in MoveIt! and the simulation is done in \texttt{trep}.}
\label{fig-baxter_y}
\end{figure}

The purpose of this work is to use a robotic arm to first identify the behavior of a flexible object and second plan how to maneuver the object to a desired location or configuration. This is an important step toward manipulation of flexible objects such as rubber tubes, plants or clothes \cite{wakamatsu2006knotting,saha2007manipulation,bell2010flexible,jimenez2012survey}.  

There are many methods to model and simulate flexible objects \cite{khalil_payeur,lang_etal}.  A common approach is to model the object as a lattice or collection of links of masses and springs \cite{sahari_etal,wakamatsu_etal,khalil_payeur}.  This approach has been used to simulate linear object like strings, hair, and electrical cables for which the model is a series of masses linked together with springs. %%% MISSING: LIMITATIONS OF THIS APPROACH  

% 3. What are you doing that is new and why do you think it will be successful
The approach taken in this paper is a follow up to our work in \cite{caldwell_coleman_correll_iros}. In that paper, we identified the spring constants of a flexible loop using touch alone. In the current paper, we extend this result to also use visual perception during the identification. Furthermore, the object under consideration in this paper has a more complex geometry.  We find that identification from touch using a single robotic arm is not sufficient and as such we need additional measurements like vision.  The new object is shaped similar to the letter `Y' and was chosen in part due to its resemblance to plant life.  To the best of our knowledge, no one has used a robot to identify parameters of a flexible object with such a geometry, nor has anyone planned robotic manipulation of such a flexible object.

In order to identy parameters of the object, such as its stiffness profile, we assume the robot has rigidly grasped the object at one end. The robot then bends, twists, and stretches the object.  During the manipulation, the robot measures the arm's joint torques and joint angles.  Furthermore, a vision system captures the movement of the full object in three dimensions.  With this information, it is possible to back out mechanical properties of the loop in order to generate an accurate model for future control and manipulation. We model the object with the same underlying mechanics as the robot arm---i.e., as a collection of rigid bodies connected by springs---allowing us to utilize the vast theory of rigid body mechanics \cite{murray_li_sastry}. Also, this enables planning and control to be done in the combined arm and object configuration space instead of only the end effector space or object space.  We then use an optimal control approach for calculating model properties that best match the behavior of the flexible object. The physical experiment, the vision capture, and the model can be viewed in Figure \ref{fig-baxter_y}.

As in \cite{caldwell_coleman_correll_iros}, we use variational integrators to simulate the robot and object.  Variational integrators can be used to describe discrete-time equations of motion of a mechanical system.  They are designed from the least action principle and have good properties that agree with known physical phenomenon like stable energy behavior \cite{pekarek_murphey}. All simulations were implemented in \texttt{trep} \cite{johnson_murphey_scalable,johnson_murphey_linearization}, which is a tool that simulates articulated rigid bodies using midpoint variational integrators.

Finally, we demonstrate how to use optimal controls to plan a trajectory---i.e. calculate configurations and controls---that maneuvers the object from one configuration to another. Furthermore, we use this result along with a feedback loop to plan a trajectory that moves a point on the flexible object near a desired location in space.

\subsection{Organization of this paper}
This paper is organized as follows: Section \ref{sec-exp} sets up the experiment with the robot and flexible object. In Section \ref{sec-sim}, the flexible object is modeled as a connection of springs and masses. Section \ref{sec-id} also reviews variational integrators. Section \ref{sec-vis} discusses the visual perception system and the techniques to filter and model fit the measured data. Section \ref{sec-id} reviews the identification algorithm from \cite{caldwell_coleman_correll_iros}. Section \ref{sec-id_eg} conducts the identification on the flexible object and compares identification with and without vision. Finally, Section \ref{sec-plan} provides the details on applying optimal control for planning a trajectory for the flexible object.

\section{Example Experiment Setup}
\label{sec-exp}
\begin{figure}[!htb]
\centering
\def\svgwidth{.6\textwidth}%500pt}
\input{baxter_y_sim.pdf_tex}
\caption{Model of Baxter manipulating a flexible object. The object is composed of 3 tubes, each approximated with four rigid links. The spheres illustrate the location of the masses and their relative values. }
\label{fig-baxter_y_sim}
\end{figure}

The goal of the example experiment is 1) to identify dynamic properties of a flexible obect and 2) to use the identified model for trajectory planning. The flexible object is shaped like the letter `Y'.  It is supposed to resemble the basic growth geometry of a living plant. The base, or `trunk' is attached to the ground and is labelled $T_1$. The robot has the end of one branch $T_2$ grasped, and the free branch is $T_3$ (refer to Figure \ref{fig-baxter_y_sim}.  The goal is to manipulate the grasped branch so that the movement of the uncontrolled free branch can be predicted for planning.  
% branch enters the work space of the end effector of the free arm. 
Due to the flexibility of the object it is not immediately clear how to do this planning.
% a manipulation that moves the free branch to the desired region. 
For this reason, we first conduct the parameter identification from \cite{caldwell_coleman_correll_iros} in order to identify a model of the flexible object. We apply the identification using only physical contact data---i.e. joint angle and torques from an arm---and will find that touch alone is not sufficient to identify the parameters. Therefore, we use a vision system to capture the motion of the full flexible object. The identification process is the same as in \cite{caldwell_coleman_correll_iros} once the vision data has been fitted to the model.  Finally, we use the optimal control technique from \cite{hauser} to calculate a trajectory to move the flexible object from one configuration to another.

The flexible object is foam tubing with differing widths for each of the $T_1$, $T_2$, and $T_3$ sections. The three tubes are connected with a `Y' PVC joint and glue.  The trunk tube has length 0.813m and mass 0.064Kg. It is the widest tube with radius 0.032m. The grasped tube has length 0.610m and mass 0.015Kg and is the second widest tube with radius 0.025m. The free tube has length 0.737m and mass 0.016Kg. It is the thinnest with radius 0.019m.  The tube widths are mentioned because we expect the relative twisting stiffnesses of each tube to correlate with their widths. 

We use Rethink Robotics' Baxter \cite{guizzo2011rethink} robot to both manipulate and measure the loop.  Each of Baxter's arms have 7 degrees of freedom.  The arms are designed for compliance by means of series elastic actuators in each joint that allows for force sensing and control.  Baxter publishes the joint angles and torques at 100Hz.  A picture of Baxter manipulating the flexible object is in Figure \ref{fig-baxter_y}. 

\section{Model and Simulation \label{sec-sim}}
\label{sec-sim}
We model the flexible object as a spring mass system. We approximate the $T_1$, $T_2$, and $T_3$ sections each with four rigid links of uniform lengths and masses, connected by joints with torsional springs---see Figure \ref{fig-baxter_y_sim}. Each joint is 3-dimensional allowing for bending and twisting motions of the flexible object. In total, the flexible object is 36 dimensional. The goal of the identification, Section \ref{sec-id}, is to identify the torsional springs' spring constants. We label these parameters as $\rho$.

Due to Baxter's 7 degrees of freedom arms, the system of Baxter grasping one end of the flexible object (neglecting the other arm) has a total number of 43 configuration variables. The dimensions, inertia, and other information concerning Baxter's arm can be obtained at \url{https://github.com/RethinkRobotics}.  

\subsection{Simulation}
The model for both Baxter's arm and the flexible object are a series of rigid links connected by rotational joints. As such, the dynamics of both the manipulator and the object can be handled together. We use \emph{variational integrators} to simulate the system dynamics.   Variational integrators are a discrete-time representation of the equations of motion of a mechanical system.  They are designed from the least action principle and have good properties that agree with known physical phenomenon like stable energy behavior \cite{pekarek_murphey}.  

We will always simulate the system for a finite time interval $[0,t_f]$ for the times $t_0,t_1,\ldots,t_{k_f}$ where $t_0 = 0$, $t_{k_f} = t_f$ and $k_f+1$ is the total number of discrete times in the interval. The simulation---i.e. the solving of the system dynamics---will result in a state $x_k:=x(t_k)$.  For variational integrators, the state is composed of the configuration, labelled $q_k$ for time $t_k$, as well as a term labelled $p_k$, also for time $t_k$. For systems without external forcing, $p_k$ is the conserved momentum.  For the purposes of this paper, it can simply be thought of as analagous to the time derivative of $q_k$, which is often paired with $q_k$ to make up the state.  Therefore, the state is $x_k:=[q_k,p_k]^T$. 

The literature on variational integrators \cite{marsden_west} provides a one-step mapping to update the state at the previous time $x_k$ to the next time $x_{k+1}$.  We provide a short high-level review of variational integrators.  We write the one-step mapping which constitutes the systems equations of motion as 
\begin{equation}
x_{k+1} = f(x_k,\rho,t_k).
\label{eq-fk}
\end{equation}
Here, $f$ explicitely depends on the previous state, time, and the parameters which we wish to identify. For a single simulation, though, $\rho$ remains constant.  While we write the equations of motion as an explicit equation, the equations are in fact implicit and rely on root solving to update the state. 

The equations encapsulate the system's Lagrangian, any external forcing, holonomic constraints, as well as a choice of quadrature for approximating integrals.  The function $f$ can be linearized. We write 
\begin{equation}
A_k = \frac{\partial}{\partial x_k}f(x_k,\rho,t_k) \textrm{ and } B_k = \frac{\partial}{\partial \rho}f(x_k,\rho,t_k).
\label{eq-lin_fk}
\end{equation}
The equations to calculate the linearization with respect to the state and parameters can be found in \cite{caldwell_coleman_correll_iros}. They are needed for calculating the gradient for parameter identification as part of an iterative optimization.

\subsection{Simulation of Example}
We use variational integrators to simulate Baxter manipulating the flexible object through the software tool \texttt{trep} \cite{johnson_murphey_scalable}.  The tool simulates articulated rigid bodies using midpoint variational integrators.  It additionally provides partial derivative calculations that we need for the system linearization, Eq.(\ref{eq-lin_fk}). 

A single configuration of Baxter and the flexible object, $q_k$, is given by a configuration of dimension 43. Therefore, the system's state,  $x_k = [q_k,p_k]^T$, is 86 dimensional. The discrete dynamics $f$, Eq.(\ref{eq-fk}), is given by the discrete system Lagrangian, discrete external forcing, and holonomoic constraints---see \cite{caldwell_coleman_correll_iros,marsden_west}.  The system Lagrangian is specified by the kinetic and potential energies of Baxter and the object.  External forces enter the system through the torques applied by the motors at each of Baxter's joints.  Additionally, holonomic constraints are needed to ensure that Baxter's end effector remains in contact with the object.  We chose a time step of $0.01$ seconds, which matches the broadcast frequency of Baxter. 

Nominally, the simulation will perfectly agree with Baxter's measured joint torque and angles for a given experiment. However, due to model and sensor disturbances, this will not be the case. Furthermore, since the system is unstable---i.e. small disturbances can result in large changes to trajectory---directly feeding the measured torques into the model will not result in a meaningful simulation. Therefore, the measured joint torques, labelled $\overline{F}$, and measured joint angles, labelled $\overline{b}$, must be filtered through a feedback loop. We use a simple proportional control law with gain $K$: 
\[
F_k = \overline{F}_k - K_k (b_k - \overline{b}_k),
\]
where $F$ and $b$ are the filtered joint torques and angles.  When $K$ is large, the effect the parameters have on the simulation is dominated by the control and as such, the system can not be identified. However, if $K$ is too small, the system will remain unstable and not track the measured trajectory well enough to be meaningful.  Correctly choosing $K$ for the purposes of parameter identification of unstable systems is left for future work.  For this paper, we chose $K$ from a finite horizon LQR which results in an optimal feedback gain from the model linearized around $b_{meas}$ and a quadratic cost functional \cite{anderson_moore}.  The tradeoff between tracking the joint angles or joint torques is directly represented in the quadratic cost for specifying how large $K$ is.


\section{Vision}
\label{sec-vis}
For verification and improvement of the parameter identification algorithms, the plant model is visually tracked using an out of the box depth sensor, the Asus Xtion Pro Live. The depth sensor is positioned directly in front of the robot, pointing towards Baxter and the flexible object. The processing of the acquired point clouds consists of three steps: filtering, segmentation, and graph creation. 

A filtering component within the motion planning framework MoveIt! \cite{coleman_etal_barrier} is used to perform the first step of self-filtering for the point cloud data $G$, removing points detected on the robot’s body and arms. This is accomplished using the realtime joint states and calculated coordinate transforms to determine the robot’s configuration within the point cloud. A small amount of padding is included in the filtering to account for calibration error. The Point Cloud Library (PCL) \cite{rusu20113d} is used for the second level of filtering, removing points detected behind the robot and on the floor. Finally, a statistical outlier removal filter is used to remove remaining noise and measurement errors.

The segmentation of the filtered point cloud $G_{filtered}$ is accomplished using a custom algorithm built ontop of PCL. This step converts the $T_1$, $T_2$, and $T_3$ sections of the object into segmented components that can later be turned into a graph, as shown in Figure \textbf{TODO ADD A FIGURE}. The lowest point in the point cloud on the z-axis (near the floor) detected is used to seed the algorithm. The $k_1$ nearest neighbors to this point is then chosen using a Kd Tree to represent a “segment” $s$ of the plant model. This segment is not typically centered with the plant trunk or branch’s center, so an additional centering step is required. The centroid of $s$ is calculated and a second $k_1$ nearest neighbors search is performed to find a centered segment $s_{centered}$ at the base of the plant.

Assuming the number of points in $s_{centered}$ is above a minimum threshold (to remove noise), the 3d centroid of $s_{centered}$ is now added to a processed graph $G_{processed}$. The points in $s_{centered}$ is then removed from $G_{filtered}$. The next nearest neighbor to $s_{centered}$ is chosen as the new segment starting point and the algorithm repeats. Occasionally, there will be insufficient nearest neighbors in a segment if, for example, the algorithm has reached the end of one of the branches of the plant. In this case, a random point is chosen in the remaining point cloud $G_{filtered}$ to continue the search, until no further points remain.

The last vision tracking step takes the disconnected points in $G_{processed}$ and performs one final series of nearest neighbor searches to connect the nodes together to represent a flexible object modeled as a series of connected rigid bodies. These connected rigid bodies give a surprisingly accurate three dimensional reconstruction of a flexible object in soft-realtime, processing new point clouds at a rate averaging 3 hz.

\subsection{Fitting to Model \label{sec-fit}}

The processed points in $G_{processed}$ need to be fitted to the static model of the plant to be useful. The model is a discretization of the physical object where the flexible object's configuration specifies the location of the discretization points. Label the object’s configuration as $q_o$. 

The fitting is a calculation on $q_o$ and is accomplished as follows: Let $G_{model}(q_o)$ be a graph specified for configuration $q_o$ with the discrete points as its vertices and adjacent points in the model as its edges. Any two adjacent points in $G_{model}(q_o)$ can be connected by a line segment in space. Let $L_{model}(q_o)$ be the collection of these line segments. Further, let $d(p,\ell)$ be the shortest distance in space between the point $p\in G_{processed}$ and line segment $\ell \in L_{model}(q_o)$. Define $d(p,L_{model}(q_o)) := \min_{\ell\in L_{model}(q_o)} d(p,\ell)$ as the least distance for $p$ from any line segment in $L_{model}(q_o)$. This can be done for each $p\in G_{processed}$. The fitting is given by the $q_o$ for which the points in $G_{processed}$ are nearest the line segments $L_{model}(q_o)$---i.e. by the optimization program
\begin{equation}
\arg \min_{q_0} \sum_{p\in G_{processed}} d(p,L_{model}(q_o)).
\label{eq-fit_prog}
\end{equation}

\subsection{Vision Tracking Concerns \label{sec-vis_disc}}

A number of tuned parameters make the vision filtering and fitting algorithms sensitive to object size, the distance between the camera and object, and variability of the object's thicknesses. However, it works well for our experimental goals.

One major shortcoming of the vision tracking pipeline developed for this experiment is occasional choppiness and buffering issues. Because of the computational complexity of our flexible object manipulation pipeline, the experiment was run on 3 distributed commodity PCs using ROS \cite{quigley2009ros}. A common issue was clock time synchronization between the three PCs, and ROS messages being dropped because of full buffers. This caused the robot transforms to be published with old time stamps and the robot self-filtering of the point clouds to stall, ultimately resulting in choppiness in the visual tracking of the plant model. This is an area of continued investigation and improvement. 

\section{Identification}
\label{sec-id}
The goal of the identification is to calculate model parameters of the system that best agree with the physical behavior of the physical object. For the example in the paper, the parameters we wish to identify are the spring constants of the flexible object spring mass model. 

Each joint of the flexible object is 3-dimensional: the joint can rotate along each axis. As seen in figure \ref{fig-tube_link}, each joint frame has the $Z$-axis aligned with the link. Therefore, a bend in the tube at a joint is realized by a rotation about the $X$- and $Y$-axes and a twist in the tube is a rotation about the $Z$-axis. Because the foam is uniformly distributed for each tube, we assume that the spring constants associated with bending---i.e. rotations about the $X$ and $Y$ axes---are the same for a single tube. 

Label the torsional spring constant about the $X$-axis (alternatively $Y$ or $Z$) for the $i^{\textrm{th}}$ tube as $\kappa_{T_i,X}$ (alt, $\kappa_{T_i,Y}$ or $\kappa_{T_i,Z}$). There are 6 total parameters $\rho = [\rho_1,\ldots,\rho_6]$ in our model, where 
\begin{equation}
\begin{array}{l}
\rho_1 = \kappa_{T_1,X} = \kappa_{T_1,Y}, \\
\rho_2 = \kappa_{T_1,Z}, \\
\rho_3 = \kappa_{T_2,X} = \kappa_{T_2,Y}, \\
\rho_4 = \kappa_{T_2,Z} , \\
\rho_5 = \kappa_{T_3,X} = \kappa_{T_3,Y}, \textrm{ and} \\
\rho_6 = \kappa_{T_3,Z}. 
\end{array}
\label{eq-params}
\end{equation}

The goal of this section is to identify $\rho$ by finding the $\rho$ with simulation that best matches measured data.  We compare results for when the measured data is 1) only Baxter's joint torque and joint angle with 2) Baxter's joint torque, joint angles, and vision measurements.

\begin{figure}[!htb]
\centering
\def\svgwidth{.80\textwidth}%500pt}
\input{tube_link.pdf_tex}
\caption{Illustration of joint $i$ connecting links $i$ and $i+1$ of the flexible object.  The joint is a rotation around the $X$, $Y$, and $Z$ axes.}
\label{fig-tube_link}
\end{figure}

\subsection{Optimal Parameter Identification \label{sec-opt}}
The goal of parameter optimization is to calculate the model parameters $\rho$ that minimize a cost functional.  The cost functional is the integral of a running cost $\ell_d(x_k,\rho)$ plus a terminal cost $m(x_{k_f},\rho)$:
\[
\min_{\rho} \Big[J_d(\rho):=\sum_{k=1}^{k_f}\ell_d(x_k,\rho) + m_d(x_{k_f},\rho)\Big]
\]
constrained to the dynamics, $x_{k+1} = f(x_k,\rho,t_k)$. Since this is a nonlinear optimal control problem, we turn to iterative methods like steepest descent to calculate a local minima. In order to apply steepest descent, we must have access to the gradient of the cost, which is given in the following Lemma from \cite{caldwell_coleman_correll_iros}. 
\begin{lemma}
\label{lem-grad_a}
Suppose $f(x_k,\rho,t_k)$ is $\mathcal{C}^2$ with respect to $x_k$ and $\rho$.  Let $A_k$ and $B_k$ form the linearization of $f$, Eq.(\ref{eq-lin_fk}), and assume $f_k$ exists.  Then,
\begin{equation}
\nabla J_d(\rho) = \sum_{k = 1}^{k_f}\lambda_kB_{k-1} +\frac{\partial}{\partial \rho}\ell_d(x_k,\rho) + \frac{\partial}{\partial \rho}m_d(x_{k_f},\rho)
\label{eq-DJa}
\end{equation}
where $\lambda_k$ is the solution to the backward one-step mapping
\begin{equation}
\lambda_k = \lambda_{k+1}A_{k} + \frac{\partial}{\partial x_{k}}\ell_d(x_{k},\rho) 
\label{eq-lambda}
\end{equation}
starting from $\lambda_{k_f} = \frac{\partial}{\partial x_{k_f}}\ell(x_{k_f},\rho) + \frac{\partial}{\partial x_{k_f}}m_d(x_{k_f},\rho)$.  
\end{lemma}

It is worth noting that $f_k$ is not guaranteed to exist, but its existance can be checked using the Implicit Function Theorem. \cite{johnson_schultz_murphey} shows a couple of scenarios where such singularities occur.  Also, \cite{miller_murphey} reports the gradient and Hessian for optimal parameter identification in continuous time.

The steepest descent direction is $-\nabla J_d(\rho)$ and the steepest descent algorithm can be applied \cite{kelley}.

\section{Identification for Example}
\label{sec-id_eg}
The parameters to be identified are the spring constants of the flexible object given by the six dimensional $\rho$.  The identification calculates the value $\rho$ with simulation that best matches the measured data. We do the matching for two sets of measured data.  The first is just Baxter's arm joint torque and joint angle measurements while the second also includes vision (see Section \ref{sec-vis}).  

\subsection{Without Vision \label{sec-no_vis}}
Without vision, the only measurements are Baxter's arm joint torques and angles.  Since Baxter is only in contact with the object at its end effector, the robot can only measure the flexible object's movement at that single point. The cost $J_d(\rho)$ for parameters $\rho$ is chosen to compare the position of Baxter's end effector in space between the simulation and the measured data.  Label $w_k(\rho)$ as this simulated end effector's position at time $t_k$ for parameters $\rho$. Furthermore, label $\overline{w}_k$ as this measured position at $t_k$, calculated from the measured robot joint angles.  Set $\epsilon_k := w_k(\rho)-\overline{w}_k$.  The cost $J_d$ is defined by the running cost $\ell_d(q_k,\rho)$ and the terminal cost $m_d(q_{k_f},\rho)$, set as
\[
\ell_d(q_k,\rho) = \epsilon_k^T\epsilon_k \textrm{ and } m_d(q_{k_f},\rho) = \epsilon_{k_f}^T\epsilon_{k_f}.
\]
We calculate the locally optimal parameters $\rho^\star$ using steepest descent with gradient calculated in Section \ref{sec-id}. There is an inequality constraint that each parameter $\rho_i\geq 0$, for $i = 1,\ldots,6$. We seed the iterative optimization with an initial guess of $\rho = [3, 3, 3, 3, 3, 3]^T$.  %#At each iteration of steepest descent, an Armijo line search updates the parameters by choosing a distance to step in the direction of the negative gradient.  We used Armijo parameters $\alpha = \beta = 0.4$ \cite{armijo}.

The optimization results in $\rho^\star = [23.780,  0.000 , 18.357 , 11.103 , 3.200,  3.059]^T$. Refer to Eq.(\ref{eq-params}) to compare which parameters correspond to which spring constants. 

\emph{Remarks on the results}
\begin{itemize}
\item The spring constant $\kappa_{T_1,Z} = \rho^\star_2 = 0$ is likely due to attempting to identify with insufficiently disparate measurements. Since the only measurement location is at the tip of $T_2$, it is possible that a continuum of model parameters could have equal quality matching the movement of the single measurement location---or at least nearly equal quality up to measurement noise. As such, the calculated optimal parameters are likely a result of the choice of cost.
\item The final two parameters, $\kappa_{T_3,X} = \kappa_{T_3,Y} = \rho^\star_5 = 3.200$, and $\kappa_{T_3,Z} = \rho^\star_6 = 3.059$ are nearly the value of their initial guess to the optimization. These two parameters specify the spring constants of the free, ungrasped branch. While the free branch's movement affects the movement of the grip location, it's impact is likely insignificant compared to the movement of the rest of the object and is reflected by an insensitivity in the optimization.
\end{itemize}
Ultimately, the flexible object identification as set up here with a single point of measurement is inadequate. We turn to vision next to improve identification quality.

\subsection{With Vision \label{sec-w_vis}}
With vision, the motion of more locations on the flexible object can be measured than with touch alone. However, as discussed in Section \ref{sec-vis_disc}, our vision measurements arrive at irregular intervals and sometimes of dubious quality. However, the identification result is an improvment compared to without it. The following process filters the vision data, removing bad data and aligning the good data with the timing of the simulation:

From Section \ref{sec-vis}, each frame of vision data is processed resulting in points $G_{processed}$ and fitted to the model with optimal fit of $q_o^\star$. Recall $q_o^\star$ is the configuration of the object that best fits the data and is calculated from the program Eq.(\ref{eq-fit_prog}). The frame occurs at a time $s$ and so we label that frame's fit as $q_o^\star(s)$.  Furthermore, the quality of the fit is quantified by the value of $d(p,L_{model}(q_o^\star(s)))$ where lesser values correspond to better fits.  As such, ``good'' data is when $\sum_{p\in G_{processed}}d(p,L_{model}(q_o^\star(s)))<d_{max}$, a user specified tolerance. Data that does not meet this requirement is discarded.  In order to align the data timing with the simulation, we first interpolate in time over the remaining data using a cubic spline, and label the result $q_{o,interp}^\star$.  Second, we calculate the simulation times $t_k$ as close as possible to the timing $s$ of the remaining data. Define $\sigma = \{\sigma_1,\ldots,\sigma_{k_f}\}$ as $\sigma_k = 1$ if $t_k$ is the simulation time nearest a vision frame time $s$.  In the identification, the cost function depends on the vision data $q_{o,interp}^\star(t_k)$ for $\sigma_k = 1$.

In order to conduct the identification using the optimization method in Section \ref{sec-id}, we need to specify a cost function.  The cost depends on Baxter's joint angles in the same manner as in Section \ref{sec-no_vis}, where the only point of measurement on the flexible object is the grasp location at the robot's end effector. With vision, we use the vision fitted configurations, $q_{o,interp}^\star$, to also compare to the simulation the measured locations of the flexible object's juncture point and ungrasped branch end point (see Figure \ref{fig-baxter_y_sim} for these two points). Label $w_k^{ee}(\rho)$, $w_k^j (\rho)$, and $w_k^{bt}(\rho)$ as the simulated end effector point, joint point, and ungrasped branch tip point for parameters $\rho$ at time $t_k$, respectively. Similarily, label the measured points as $\overline{w}_k^{ee}$, $\overline{w}_k^j$, and $\overline{w}_k^{bt}$. The measured data $\overline{w}_k^{j}$ and $\overline{w}_k^{bt}$ is calculated from $q_{o,interp}^\star$ and is only valid for times $t_k$ where $\sigma_k = 1$.  

The cost for identification depends on the error between the simulation and the measured. Set
\[
\epsilon^{ee}_k := w^{ee}_k(\rho)-\overline{w}^{ee}_k \textrm{, } \epsilon^{j}_k = \sigma_k(w^j_k(\rho)-\overline{w}^j_k) \textrm{, and } \epsilon^{j}_k = \sigma_k(w^{bt}_k(\rho)-\overline{w}^{bt}_k).
\]
The running cost is 
\[
\ell_d(q_k,\rho) = (\epsilon^{ee}_k)^TQ^{ee}\epsilon^{ee}_k + (\epsilon^{j}_k)^TQ^{j}\epsilon^{j}_k + (\epsilon^{bt}_k)^TQ^{bt}\epsilon^{bt}_k
\]
where the weights are $Q^{ee} = Q^{j} = Q^{bt} = Id_{3}$ and $Id_{3}$ is the $3\times3$ identity matrix.  We set the running cost to be $m_d(q_{k_f},\rho) = \ell_d(q_{k_f},\rho)$.

We execute the parameter identification as in Section \ref{sec-no_vis} starting with an initial guess of $\rho = [3, 3, 3, 3, 3, 3]^T$. The optimization results in identified parameters: $\rho = [22.270 , 13.593 , 10.735 ,  8.202  ,11.5111,   9.692]^T$.

The stiffest tube is $T_1$, where $\kappa_{T_1,X} = \kappa_{T_1,Y} = \rho^\star_1$ and $\kappa_{T_1,Z} = \rho^\star_2$ have the greatest spring constants for each respective axis. This result is expected since $T_1$ is the thickest tube. It turns out that the second stiffest tube is $T_3$, which is the thinnest. It is likely that this result is incorrect and is due to $T_3$ not being directly manipulated. Throughout the motion, the other two tubes were bent and twisted significantly more than $T_3$ and as such the identification of $T_3$ should not be held to a high level of confidence.  In other words, while additional locations of measurement increased the quality of identification, additional locations of manipulation are needed for greater confidence.

\section{Planning}
\label{sec-plan}
\begin{enumerate}
\item Goal: calculate torques and state trajectory that moves the other end of the flexible-Y to a desired point in space.
\item Optimal control stuff.
\item Results.
\end{enumerate}

\section{Discussion / Conclusion}
\label{sec-conc}


\bibliographystyle{plain}
\bibliography{ISER2014_cites}

\end{document}
